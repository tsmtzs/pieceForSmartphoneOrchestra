\documentclass[11pt,letterpage]{book}
\usepackage{xunicode}
\usepackage{xltxtra}
\usepackage{xgreek}

\setmainfont[Mapping=tex-text]{Linux Biolinum O}

\usepackage[centering]{geometry}
\usepackage{color}
\usepackage{graphicx}
\usepackage{tikz}
\usetikzlibrary{calc,decorations.pathmorphing}

\pagestyle{empty}

\title{Piece for smartphone orchestra}
\author{Tassos Tsesmetzis}
\date{}

\begin{document}
\thispagestyle{empty}
\maketitle

\newpage

\chapter*{ Κομμάτι για ορχήστρα smartphone}
\section*{Οδηγίες}
\begin{itemize}
\item Δε χρειάζεται να παίζεις συνεχώς.
\item Ξεκίνα βαθμιαία.
\item Σταμάτα βαθμιαία.
\item Ενσωματώσου στο συνολικό ήχο.
\item Επεδίωξε τη συνέχεια του συνολικού ήχου.
\item Αν θέλεις, πάρε τον ήχο από κάποιο γειτονικό εκτελεστή.
\item Πριν σταματήσεις, μπορείς να δώσεις τον ήχο σου σε κάποιο διπλανό εκτελεστή.
\item Διατήρησε την ισορροπία.
\item Όταν η στιγμή είναι κατάλληλη, δημιούργησε διακρoτήματα.
\item Μπορείς να κινείσαι στο χώρο.
\item Ακολούθησε το σχήμα\\
  \begin{center}
    \begin{tikzpicture}
      [%
      line/.style={line width=0.1pt}
      ]
      \coordinate (A) at (0,0);
      \coordinate (B) at (1,0);
      \coordinate (C) at (2,0.25);
      \coordinate (D) at (3,0.5);
      \coordinate (E) at (4,0.75);
      \coordinate (F) at (5,1.0);
      \coordinate (G) at (6,0.75);
      \coordinate (H) at (7,0.5);
      \coordinate (I) at (8,0.25);
      \coordinate (J) at (9,0);
      \coordinate (K) at (10,0);
      \coordinate (L) at (11,0);
      \coordinate (M) at (12,0);
      \coordinate (N) at (13,0);
      \foreach \letter in {A,B,...,N} {%
        \fill (\letter) circle [radius=0.025cm];
        \fill (\letter |- {{(0,0)}}) circle [radius=0.025cm];
      }
      % \draw [line, out=60, in=-160] (0,0) to (2,0);
      % \draw [line, decorate, decoration={random steps,segment length=0.2cm, amplitude=0.1cm}] (0,0) -- (2,1);
      % \draw [line, decorate, decoration={random steps,segment length=0.2cm, amplitude=0.1cm},out=30, in=-49] (0,0) to (2,1);
      % \draw [line, decorate, decoration={snake,segment length=0.8cm},out=50, in=170] (0,0) to (2,-1);
      \draw [line, decorate, decoration={aspect,amplitude=0.5cm},out=-10, in=-180] (0,0) to (2,-0.5);
    \end{tikzpicture}
  \end{center}
\item Σταμάτα να παίζεις όταν θεωρήσεις ότι επιτεύχθηκε ο σκοπός\footnote{\textit{Σκοπός}: Μια νοητική εικόνα για τη μορφή που πρέπει να έχει το έργο.
    Πρόκειται για ένα δίκτυο νοητικών χώρων που διαμορφώνονται από τις οδηγίες, τις πρόβες και τις διάφορες εκτελέσεις του κομματιού.}
  και το κομμάτι μπορεί να τελειώσει.
\end{itemize}
\section*{Παρατηρήσεις}
\begin{itemize}
\item Οι εκτελεστές βρίσκονται ανάμεσα και γύρω από το ακροατήριο.
\item Καλύτερος ήχος μπορεί να επιτευχθεί αν χρησιμοποιηθούν φορητά ηχεία.
\item Οι εκτελεστές πρέπει να είναι τουλάχιστον πέντε.
\item Υπάρχει μια εγγενής θεατρικότητα στις κινήσεις των εκτελεστών.
\item Η μεταφορά ενός ήχου από έναν εκτελεστή σε άλλο μπορεί να επιτευχθεί με οπτική ή νευματική επικοινωνία.
\item Οι εκτελεστές έχουν καταφορτώσει την ιστοσελίδα του έργου πριν την έναρξη. Επίσης, πατώντας στον τίτλο του έργου η συσκευή μεταβαίνει σε κατάσταση πλήρους οθόνης.
\end{itemize}
{\footnotesize\phantom{}\hfill Ξάνθη, 2018 -- 2020}

\newpage

\chapter*{Piece for smartphone orchestra}
\section*{Directions}
\begin{itemize}
\item You don't have to play all the time.
\item Start playing gradually.
\item Stop playing gradually.
\item Absorb in the global sound.
\item Strive for the continuity of the general sound.
\item You can take the sound of a neighboring performer, if you want.
\item Before you stop you can give your sound to a close performer.
\item Strive for balance.
\item Make beats when the moment is appropriate.
\item You can move if you like.
\item Follow the general shape\\
  \begin{center}
    \begin{tikzpicture}
      [%
      line/.style={line width=0.1pt}
      ]
      \coordinate (A) at (0,0);
      \coordinate (B) at (1,0);
      \coordinate (C) at (2,0.25);
      \coordinate (D) at (3,0.5);
      \coordinate (E) at (4,0.75);
      \coordinate (F) at (5,1.0);
      \coordinate (G) at (6,0.75);
      \coordinate (H) at (7,0.5);
      \coordinate (I) at (8,0.25);
      \coordinate (J) at (9,0);
      \coordinate (K) at (10,0);
      \coordinate (L) at (11,0);
      \coordinate (M) at (12,0);
      \coordinate (N) at (13,0);
      \foreach \letter in {A,B,...,N} {%
        \fill (\letter) circle [radius=0.025cm];
        \fill (\letter |- {{(0,0)}}) circle [radius=0.025cm];
      }
      % \draw [line, out=60, in=-160] (0,0) to (2,0);
      % \draw [line, decorate, decoration={random steps,segment length=0.2cm, amplitude=0.1cm}] (0,0) -- (2,1);
      % \draw [line, decorate, decoration={random steps,segment length=0.2cm, amplitude=0.1cm},out=30, in=-49] (0,0) to (2,1);
      % \draw [line, decorate, decoration={snake,segment length=0.8cm},out=50, in=170] (0,0) to (2,-1);
      \draw [line, decorate, decoration={aspect,amplitude=0.5cm},out=-10, in=-180] (0,0) to (2,-0.5);
    \end{tikzpicture}
  \end{center}
\item Stop your playing when you think that the goal\footnote{\textit{Goal}: A mental picture about the form of the piece. Consists of a web of mental spaces that form according to the directions, the rehearsals and the various performances of the piece.} has accomplished and the piece can now end.
\end{itemize}
\section*{Remarks}
\begin{itemize}
\item Performers are distributed around and among the audience.
\item Portable speakers can be used for better sound quality.
\item The number of the performers should be greater or equak to five.
\item There is an inherent theatrical element in the movements of the performers.
\item Distribution of sound between performers can be achieved by using visual signs  or gestures.
\item Performers download the web page of the piece before the performance starts. Fullscreen mode can be enabled by cicking on the title.
\end{itemize}
{\footnotesize\phantom{}\hfill Xanthi, 2018 -- 2020}
\end{document}