\documentclass[11pt,letterpage]{book}
\usepackage{xunicode}
\usepackage{xltxtra}
\usepackage{xgreek}

\setmainfont[Mapping=tex-text]{Linux Biolinum O}

\usepackage[centering]{geometry}
\usepackage{color}
\usepackage{graphicx}
\usepackage{tikz}
\usetikzlibrary{calc,math,shadings,decorations.pathmorphing}

\pagestyle{empty}

\title{\Huge Piece for smartphone orchestra}
\author{Tassos Tsesmetzis}
\date{}

% Drawing
\newcommand{\graphicScore}[2]{
  \begin{center}
    \begin{tikzpicture}
      [%
      line/.style={line width=0.1pt},
      ]
      \tikzmath{
        \outAB = 5;
        \inAB = 200;
        \xB = 1;
        \yB = 0.1;
        \xC = 3;
        \yC = 1.0;
        \xD = 5;
        \yD = 0.25;
        \outDE = -18;
        \inDE = 177;
      }
      \coordinate (A) at (0,0);
      \coordinate (B) at (\xB,\yB);
      \coordinate (B') at (\xB,-1*\yB);
      \coordinate (C) at (\xC,\yC);
      \coordinate (C') at (\xC,-\yC);
      \coordinate (D) at (\xD,\yD);
      \coordinate (D') at (\xD,-\yD);
      \coordinate (E) at (7.5,0);
      % draw shape
      \node [left] at (A) {\scriptsize #1};
      \node [right] at (E) {\scriptsize #2};
      \fill [fill=black!100,shading=radial,inner color=black!80,outer color=black!15] (A) to [out=\outAB,in=\inAB] (B) .. controls (C) .. (D) %
      to [out=\outDE,in=\inDE] (E) %
      to [out=-\inDE,in=-\outDE] (D')%
      .. controls (C') .. (B') to [out=-\inAB,in=-\outAB] (A);
    \end{tikzpicture}
  \end{center}
}

\newlength{\groupDistance}
\setlength{\groupDistance}{-8pt}

\begin{document}
\thispagestyle{empty}
\maketitle

\newpage

\chapter*{Piece for smartphone orchestra}
\section*{Directions}
\begin{itemize}
\item You don't have to play all the time.
\item Start a tone gradually.
\item Stop a tone gradually.\\[\groupDistance]
\item Absorb in the global sound.
\item Strive for the continuity of the general sound.
\item Strive for balance.\\[\groupDistance]
\item Give rise to beats when appropriate.
\item Try to trigger auditory distortion products, when the context is appropriate.\\[\groupDistance]
\item You can take the sound of a neighboring performer if you want.
\item You can change your position if you like.\\[\groupDistance]
\item Follow the general shape\\
  \graphicScore{START}{END}
  \vspace{\groupDistance}
\item Stop your playing when you think that the goal\footnote{\textit{Goal}: A mental picture about the form of the piece. Consists of a web of mental spaces that form according to the directions, the rehearsals and the various performances of the piece.} has accomplished and the piece can now end.
\end{itemize}
\section*{Remarks}
\begin{itemize}
\item Performers are distributed among and around the audience.
\item Portable speakers may be used for better sound quality.
\item There is an inherent theatrical element in the movements of the performers.
\item Performers download the web page of the piece before the performance.
\end{itemize}

\vspace{1ex}

{\footnotesize\strut\hfill Xanthi, 2018--2020}

\newpage

% Reset footnote counter to 1
\setcounter{footnote}{0}

\chapter*{Κομμάτι για ορχήστρα smartphone}
\section*{Οδηγίες}
\begin{itemize}
\item Δε χρειάζεται να παίζεις συνεχώς.
\item Ξεκίνα έναν ήχο βαθμιαία.
\item Σταμάτα έναν ήχο βαθμιαία.\\[\groupDistance]
\item Ενσωματώσου στο συνολικό ήχο.
\item Επεδίωξε τη συνέχεια του συνολικού ήχου.
\item Διατήρησε την ισορροπία.\\[\groupDistance]
\item Όταν η στιγμή είναι κατάλληλη, γίνε το έναυσμα για διακροτήματα.
\item Όταν το πλαίσιο είναι κατάλληλο, προσπάθησε να πυροδοτήσεις ωτοακουστικές εκπομπές προϊόντων ακουστικής παραμόρφωσης.\\[\groupDistance]
\item Πάρε τον ήχο κάποιου γειτονικού εκτελεστή, αν θέλεις.
\item Μπορείς να μετακινείσαι στο χώρο, αν θέλεις.\\[\groupDistance]
\item Ακολούθησε το σχήμα\\
  \graphicScore{ΑΡΧΗ}{ΤΕΛΟΣ}
  \vspace{\groupDistance}
\item Σταμάτα να παίζεις όταν θεωρήσεις ότι επιτεύχθηκε ο σκοπός\footnote{\textit{Σκοπός}: Μια νοητική εικόνα για τη μορφή που πρέπει να έχει το έργο.
    Πρόκειται για ένα δίκτυο νοητικών χώρων που διαμορφώνονται  από τις οδηγίες, τις πρόβες και τις διάφορες εκτελέσεις του κομματιού.}
  και το κομμάτι μπορεί να τελειώσει.
\end{itemize}
\section*{Παρατηρήσεις}
\begin{itemize}
\item Οι εκτελεστές βρίσκονται ανάμεσα και γύρω από το ακροατήριο.
\item Καλύτερος ήχος μπορεί να επιτευχθεί αν χρησιμοποιηθούν φορητά ηχεία.
\item Υπάρχει μια εγγενής θεατρικότητα στις κινήσεις των εκτελεστών.
\item Οι εκτελεστές έχουν καταφορτώσει την ιστοσελίδα του έργου πριν την έναρξη.
\end{itemize}

\vspace{1ex}

{\footnotesize\strut\hfill Ξάνθη, 2018--2020}
\end{document}